%----------------------------------------------------------------------------------------
%	CONFIGURATIONS
%----------------------------------------------------------------------------------------

\documentclass[12pt,a4paper,oneside]{article}

\usepackage[utf8]{inputenc}
\usepackage{graphicx}
\usepackage{epstopdf}
\usepackage{natbib}
\usepackage{amsmath}
\usepackage{lipsum}
\usepackage{caption}
\usepackage{subcaption}
\usepackage[a4paper,left=2cm,right=2cm,top=2.5cm,bottom=2.5cm]{geometry}

%----------------------------------------------------------------------------------------
%	INFORMATION
%----------------------------------------------------------------------------------------

\title{Árvores de Decisão em Problemas de Classificação\\
  \vspace{0.1in}
  \large{Inteligência Artificial - Trabalho 4}
}

\author{Filipe Figueiredo\footnote{Filipe Figueiredo - 201203559},
  Pedro Paredes\footnote{Pedro Paredes - 201205725} e Tiago
  Castanheira\footnote{Tiago Castanheira - 201207833}, DCC - FCUP}

\date{Junho 2015}

\renewcommand{\tablename}{Tabela}
\renewcommand{\figurename}{Figura}
\renewcommand{\refname}{Referências}

\begin{document}

\maketitle

%----------------------------------------------------------------------------------------
%	SECTION 1
%----------------------------------------------------------------------------------------

\section{Introdução}
\label{sec:intro}

A área de aprendizagem de máquina é uma área cada vez mais estudada e
usada em mais contextos. Além de ser uma abordagem com imenso
potencial para diferentes tipos de problemas, permite soluções muito
flexíveis. É por estas razões que cada vez mais é uma área muito
explorada por grande empresas como \cite{google:2015}
\cite{facebook:2015} e da qual dependem em vários dos seus produtos.

Sendo a aprendizagem de máquina um tema tão global, é naturalmente
dividido em vários subtemas que abordam diferentes tipos de
problemas. É nestes subtemas que aparecem os problemas de
classificação. Neste relatório exploramos este tipo de problemas,
assim como uma técnica em concreto - árvores de decisão - para a sua
solução, além de algumas das suas propriedades.

O resto do relatório está organizado da seguinte forma. Na Secção
\ref{sec:cla} introduzem-se problemas de classificação. Na Secção
\ref{sec:alg} discutem-se algumas abordagens a problemas de
classificação, em particular a das árvores de decisão. Na Secção
\ref{sec:imp} descrevem-se os detalhes da implementação. Na Secção
\ref{sec:res} apresentam-se os resultados obtidos para os conjuntos de
dados fornecidos. Finalmente na Secção \ref{sec:con} fazem-se algumas
notas finais.

%----------------------------------------------------------------------------------------
%	SECTION 2
%----------------------------------------------------------------------------------------

\section{Problemas de Classificação}
\label{sec:cla}

A classificação é um dos temas mais simples, mas também mais
importantes da aprendizagem de máquina. É um tipo de aprendizagem
supervisionada, pois necessita de instâncias já corretamente
classificadas como \textit{input}.

Resumindo o problema, é dado um conjunto de atributos, onde cada
atributo pode ter um conjunto de valores diferentes. Estes valores
podem ser discretos (atributos nominais) ou contínuos (atributos
numerais). Adicionalmente, é dada uma instância destes atributos, ou
seja, um vetor de valores correspondentes a cada atributo (como uma
tabela) que pertence potencialmente a uma de um conjunto de classes. O
objetivo do problema é identificar a classe pertence esta instância,
sabendo as classes corretas para um conjunto de instâncias dado. Isto
é, o objetivo é identificar a classe de uma observação dado um
conjunto de observações para as quais se conhecem a classe correta.

Um exemplo clássico de um problema deste tipo é o problema de
classificar um \textit{email} como SPAM ou não SPAM. Neste problema em
específico é comum aplicarem-se alguns conceitos de extração de
informação além dos vários algoritmos conhecidos de
classificação. Outro problema clássico é o problema de diagnostico
médico. Dado um conjunto de sintomas ou dados sobre um paciente,
classificar esse paciente como portador ou não de um certa
doença. Claro que há muitos mais exemplos (alguns que serão referidos
nas secções seguintes deste relatório).

%----------------------------------------------------------------------------------------
%	SECTION 3
%----------------------------------------------------------------------------------------

\section{Algoritmos para Classificação}
\label{sec:alg}

Há variadas abordagens para problemas de classificação, que se aplicam
melhor dependendo do tipo de dados, do número de atributos, dos
domínios dos valores, \ldots

Um dos algoritmos mais simples, mas ao mesmo tempo muito poderoso é o
\textit{Naive Bayes}. O seu funcionamento é baseado num uso
inteligente do teorema de \textit{Bayes} assumindo ainda,
\textit{naively}, que cada atributo é independente entre si. Isto dá
origem a um algoritmo muito escalável por ser linear que consegue ser
muito competitivo em certas áreas com algoritmos mais sofisticados. Um
dos exemplos mais antigos de sucesso com este método é o problema de
classificação de \textit{emails} como SPAM ou não SPAM. O algoritmo de
classificação é extremamente simples, mas obtém resultados muito bons
\cite{sahami:1998}.

Outro método simples é o método da regressão logística.

\subsection{Árvores de Decisão}

\lipsum[0]

\lipsum[1]

\lipsum[2]

\lipsum[3]

\lipsum[4]

\lipsum[5]


%----------------------------------------------------------------------------------------
%	SECTION 4
%----------------------------------------------------------------------------------------

\section{Implementação}
\label{sec:imp}

\lipsum[0]

\lipsum[1]

\lipsum[2]

%----------------------------------------------------------------------------------------
%	SECTION 5
%----------------------------------------------------------------------------------------

\section{Resultados}
\label{sec:res}

\lipsum[0]

\lipsum[1]

\lipsum[2]

\lipsum[3]

\lipsum[4]

\lipsum[5]


%----------------------------------------------------------------------------------------
%	SECTION 6
%----------------------------------------------------------------------------------------

\section{Conclusão}
\label{sec:con}

\lipsum[0]

\lipsum[1]

\lipsum[2]


\bibliographystyle{plain}
\bibliography{lab_report_4}

\end{document}
